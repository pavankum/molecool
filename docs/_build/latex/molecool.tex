%% Generated by Sphinx.
\def\sphinxdocclass{report}
\documentclass[letterpaper,10pt,english]{sphinxmanual}
\ifdefined\pdfpxdimen
   \let\sphinxpxdimen\pdfpxdimen\else\newdimen\sphinxpxdimen
\fi \sphinxpxdimen=.75bp\relax

\PassOptionsToPackage{warn}{textcomp}
\usepackage[utf8]{inputenc}
\ifdefined\DeclareUnicodeCharacter
% support both utf8 and utf8x syntaxes
  \ifdefined\DeclareUnicodeCharacterAsOptional
    \def\sphinxDUC#1{\DeclareUnicodeCharacter{"#1}}
  \else
    \let\sphinxDUC\DeclareUnicodeCharacter
  \fi
  \sphinxDUC{00A0}{\nobreakspace}
  \sphinxDUC{2500}{\sphinxunichar{2500}}
  \sphinxDUC{2502}{\sphinxunichar{2502}}
  \sphinxDUC{2514}{\sphinxunichar{2514}}
  \sphinxDUC{251C}{\sphinxunichar{251C}}
  \sphinxDUC{2572}{\textbackslash}
\fi
\usepackage{cmap}
\usepackage[T1]{fontenc}
\usepackage{amsmath,amssymb,amstext}
\usepackage{babel}



\usepackage{times}
\expandafter\ifx\csname T@LGR\endcsname\relax
\else
% LGR was declared as font encoding
  \substitutefont{LGR}{\rmdefault}{cmr}
  \substitutefont{LGR}{\sfdefault}{cmss}
  \substitutefont{LGR}{\ttdefault}{cmtt}
\fi
\expandafter\ifx\csname T@X2\endcsname\relax
  \expandafter\ifx\csname T@T2A\endcsname\relax
  \else
  % T2A was declared as font encoding
    \substitutefont{T2A}{\rmdefault}{cmr}
    \substitutefont{T2A}{\sfdefault}{cmss}
    \substitutefont{T2A}{\ttdefault}{cmtt}
  \fi
\else
% X2 was declared as font encoding
  \substitutefont{X2}{\rmdefault}{cmr}
  \substitutefont{X2}{\sfdefault}{cmss}
  \substitutefont{X2}{\ttdefault}{cmtt}
\fi


\usepackage[Bjarne]{fncychap}
\usepackage{sphinx}

\fvset{fontsize=\small}
\usepackage{geometry}


% Include hyperref last.
\usepackage{hyperref}
% Fix anchor placement for figures with captions.
\usepackage{hypcap}% it must be loaded after hyperref.
% Set up styles of URL: it should be placed after hyperref.
\urlstyle{same}
\addto\captionsenglish{\renewcommand{\contentsname}{Contents:}}

\usepackage{sphinxmessages}
\setcounter{tocdepth}{1}



\title{molecool Documentation}
\date{Jan 26, 2020}
\release{}
\author{molecool}
\newcommand{\sphinxlogo}{\vbox{}}
\renewcommand{\releasename}{}
\makeindex
\begin{document}

\pagestyle{empty}
\sphinxmaketitle
\pagestyle{plain}
\sphinxtableofcontents
\pagestyle{normal}
\phantomsection\label{\detokenize{index::doc}}


molecool is a python package for geometry analysis and visualization.

The file formats supported are ‘xyz’ and ‘pdb’.


\chapter{Getting Started}
\label{\detokenize{getting_started:getting-started}}\label{\detokenize{getting_started::doc}}
This page details how to get started with molecool.


\section{Installation}
\label{\detokenize{getting_started:installation}}
Download from github and install using

\sphinxtitleref{pip install \sphinxhyphen{}e .}


\subsection{Usage}
\label{\detokenize{getting_started:usage}}
Here is an example.:

\begin{sphinxVerbatim}[commandchars=\\\{\}]
\PYG{k+kn}{import} \PYG{n+nn}{molecool}

\PYG{n}{benzene\PYGZus{}symbols}\PYG{p}{,} \PYG{n}{benzene\PYGZus{}coordinates} \PYG{o}{=} \PYG{n}{molecool}\PYG{o}{.}\PYG{n}{open\PYGZus{}xyz}\PYG{p}{(}\PYG{l+s+s1}{\PYGZsq{}}\PYG{l+s+s1}{benzene.xyz}\PYG{l+s+s1}{\PYGZsq{}}\PYG{p}{)}
\end{sphinxVerbatim}


\chapter{About}
\label{\detokenize{about:about}}\label{\detokenize{about::doc}}
molecool is a python package created to analyze and manipulate the geometry files. Created during molssi 2020 best practices for software bootcamp.


\chapter{API Documentation}
\label{\detokenize{api:api-documentation}}\label{\detokenize{api::doc}}

\begin{savenotes}\sphinxatlongtablestart\begin{longtable}[c]{\X{1}{2}\X{1}{2}}
\hline

\endfirsthead

\multicolumn{2}{c}%
{\makebox[0pt]{\sphinxtablecontinued{\tablename\ \thetable{} \textendash{} continued from previous page}}}\\
\hline

\endhead

\hline
\multicolumn{2}{r}{\makebox[0pt][r]{\sphinxtablecontinued{Continued on next page}}}\\
\endfoot

\endlastfoot

{\hyperref[\detokenize{autosummary/molecool.canvas:molecool.canvas}]{\sphinxcrossref{\sphinxcode{\sphinxupquote{molecool.canvas}}}}}({[}with\_attribution{]})
&
Placeholder function to show example docstring (NumPy format)
\\
\hline
{\hyperref[\detokenize{autosummary/molecool.calculate_center_of_mass:molecool.calculate_center_of_mass}]{\sphinxcrossref{\sphinxcode{\sphinxupquote{molecool.calculate\_center\_of\_mass}}}}}(symbols, …)
&
Calculate the center of mass of a molecule.
\\
\hline
{\hyperref[\detokenize{autosummary/molecool.calculate_distance:molecool.calculate_distance}]{\sphinxcrossref{\sphinxcode{\sphinxupquote{molecool.calculate\_distance}}}}}(r\_A, r\_B)
&
Calculate the distance between two points given as numpy arrays.
\\
\hline
{\hyperref[\detokenize{autosummary/molecool.calculate_angle:molecool.calculate_angle}]{\sphinxcrossref{\sphinxcode{\sphinxupquote{molecool.calculate\_angle}}}}}(r\_A, r\_B, r\_C{[}, …{]})
&

\\
\hline
{\hyperref[\detokenize{autosummary/molecool.draw_bond_histogram:molecool.draw_bond_histogram}]{\sphinxcrossref{\sphinxcode{\sphinxupquote{molecool.draw\_bond\_histogram}}}}}(bond\_list{[}, …{]})
&

\\
\hline
{\hyperref[\detokenize{autosummary/molecool.draw_molecule:molecool.draw_molecule}]{\sphinxcrossref{\sphinxcode{\sphinxupquote{molecool.draw\_molecule}}}}}(coordinates, symbols)
&

\\
\hline
\end{longtable}\sphinxatlongtableend\end{savenotes}


\section{molecool.canvas}
\label{\detokenize{autosummary/molecool.canvas:molecool-canvas}}\label{\detokenize{autosummary/molecool.canvas::doc}}\index{canvas() (in module molecool)@\spxentry{canvas()}\spxextra{in module molecool}}

\begin{fulllineitems}
\phantomsection\label{\detokenize{autosummary/molecool.canvas:molecool.canvas}}\pysiglinewithargsret{\sphinxcode{\sphinxupquote{molecool.}}\sphinxbfcode{\sphinxupquote{canvas}}}{\emph{with\_attribution=True}}{}
Placeholder function to show example docstring (NumPy format)

Replace this function and doc string for your own project
\begin{quote}\begin{description}
\item[{Parameters}] \leavevmode
\sphinxstylestrong{with\_attribution} (\sphinxstyleemphasis{bool, Optional, default: True}) \textendash{} Set whether or not to display who the quote is from

\item[{Returns}] \leavevmode
\sphinxstylestrong{quote} \textendash{} Compiled string including quote and optional attribution

\item[{Return type}] \leavevmode
str

\end{description}\end{quote}

\end{fulllineitems}



\section{molecool.calculate\_center\_of\_mass}
\label{\detokenize{autosummary/molecool.calculate_center_of_mass:molecool-calculate-center-of-mass}}\label{\detokenize{autosummary/molecool.calculate_center_of_mass::doc}}\index{calculate\_center\_of\_mass() (in module molecool)@\spxentry{calculate\_center\_of\_mass()}\spxextra{in module molecool}}

\begin{fulllineitems}
\phantomsection\label{\detokenize{autosummary/molecool.calculate_center_of_mass:molecool.calculate_center_of_mass}}\pysiglinewithargsret{\sphinxcode{\sphinxupquote{molecool.}}\sphinxbfcode{\sphinxupquote{calculate\_center\_of\_mass}}}{\emph{symbols}, \emph{coordinates}}{}
Calculate the center of mass of a molecule.

The center of mass is weighted by each atom’s weight.
\begin{quote}\begin{description}
\item[{Parameters}] \leavevmode\begin{itemize}
\item {} 
\sphinxstylestrong{symbols} (\sphinxstyleemphasis{list}) \textendash{} A list of elements for the molecule

\item {} 
\sphinxstylestrong{coordinates} (\sphinxstyleemphasis{np.ndarray}) \textendash{} The coordinates of the molecule.

\end{itemize}

\item[{Returns}] \leavevmode
\sphinxstylestrong{center\_of\_mass} \textendash{} The center of mass of the molecule.

\item[{Return type}] \leavevmode
np.ndarray

\end{description}\end{quote}
\subsubsection*{Notes}

The center of mass is calculated with the formula
\begin{equation*}
\begin{split}\vec{R}=\frac{1}{M} \sum_{i=1}^{n} m_{i}\vec{r_{}i}\end{split}
\end{equation*}
\end{fulllineitems}



\section{molecool.calculate\_distance}
\label{\detokenize{autosummary/molecool.calculate_distance:molecool-calculate-distance}}\label{\detokenize{autosummary/molecool.calculate_distance::doc}}\index{calculate\_distance() (in module molecool)@\spxentry{calculate\_distance()}\spxextra{in module molecool}}

\begin{fulllineitems}
\phantomsection\label{\detokenize{autosummary/molecool.calculate_distance:molecool.calculate_distance}}\pysiglinewithargsret{\sphinxcode{\sphinxupquote{molecool.}}\sphinxbfcode{\sphinxupquote{calculate\_distance}}}{\emph{r\_A}, \emph{r\_B}}{}
Calculate the distance between two points given as numpy arrays.
\begin{quote}\begin{description}
\item[{Parameters}] \leavevmode
\sphinxstylestrong{r\_A, r\_B} (\sphinxstyleemphasis{np.ndarray}) \textendash{} The coordinates of points A and B.

\item[{Returns}] \leavevmode
\sphinxstylestrong{distance} \textendash{} The distance between the two points A and B.

\item[{Return type}] \leavevmode
float

\end{description}\end{quote}
\subsubsection*{Examples}

\begin{sphinxVerbatim}[commandchars=\\\{\}]
\PYG{g+gp}{\PYGZgt{}\PYGZgt{}\PYGZgt{} }\PYG{n}{r1} \PYG{o}{=} \PYG{n}{np}\PYG{o}{.}\PYG{n}{array}\PYG{p}{(}\PYG{p}{[}\PYG{l+m+mi}{1}\PYG{p}{,} \PYG{l+m+mi}{0}\PYG{p}{,} \PYG{l+m+mi}{0}\PYG{p}{]}\PYG{p}{)}
\PYG{g+gp}{\PYGZgt{}\PYGZgt{}\PYGZgt{} }\PYG{n}{r2} \PYG{o}{=} \PYG{n}{np}\PYG{o}{.}\PYG{n}{array}\PYG{p}{(}\PYG{p}{[}\PYG{l+m+mi}{3}\PYG{p}{,} \PYG{l+m+mi}{0}\PYG{p}{,} \PYG{l+m+mi}{0}\PYG{p}{]}\PYG{p}{)}
\PYG{g+gp}{\PYGZgt{}\PYGZgt{}\PYGZgt{} }\PYG{n}{calculate\PYGZus{}distance}\PYG{p}{(}\PYG{n}{r1}\PYG{p}{,} \PYG{n}{r2}\PYG{p}{)}
\PYG{g+go}{2.0}
\end{sphinxVerbatim}

\end{fulllineitems}



\section{molecool.calculate\_angle}
\label{\detokenize{autosummary/molecool.calculate_angle:molecool-calculate-angle}}\label{\detokenize{autosummary/molecool.calculate_angle::doc}}\index{calculate\_angle() (in module molecool)@\spxentry{calculate\_angle()}\spxextra{in module molecool}}

\begin{fulllineitems}
\phantomsection\label{\detokenize{autosummary/molecool.calculate_angle:molecool.calculate_angle}}\pysiglinewithargsret{\sphinxcode{\sphinxupquote{molecool.}}\sphinxbfcode{\sphinxupquote{calculate\_angle}}}{\emph{r\_A}, \emph{r\_B}, \emph{r\_C}, \emph{degrees=False}}{}
\end{fulllineitems}



\section{molecool.draw\_bond\_histogram}
\label{\detokenize{autosummary/molecool.draw_bond_histogram:molecool-draw-bond-histogram}}\label{\detokenize{autosummary/molecool.draw_bond_histogram::doc}}\index{draw\_bond\_histogram() (in module molecool)@\spxentry{draw\_bond\_histogram()}\spxextra{in module molecool}}

\begin{fulllineitems}
\phantomsection\label{\detokenize{autosummary/molecool.draw_bond_histogram:molecool.draw_bond_histogram}}\pysiglinewithargsret{\sphinxcode{\sphinxupquote{molecool.}}\sphinxbfcode{\sphinxupquote{draw\_bond\_histogram}}}{\emph{bond\_list}, \emph{save\_location=None}, \emph{dpi=300}, \emph{graph\_min=0}, \emph{graph\_max=2}}{}
\end{fulllineitems}



\section{molecool.draw\_molecule}
\label{\detokenize{autosummary/molecool.draw_molecule:molecool-draw-molecule}}\label{\detokenize{autosummary/molecool.draw_molecule::doc}}\index{draw\_molecule() (in module molecool)@\spxentry{draw\_molecule()}\spxextra{in module molecool}}

\begin{fulllineitems}
\phantomsection\label{\detokenize{autosummary/molecool.draw_molecule:molecool.draw_molecule}}\pysiglinewithargsret{\sphinxcode{\sphinxupquote{molecool.}}\sphinxbfcode{\sphinxupquote{draw\_molecule}}}{\emph{coordinates}, \emph{symbols}, \emph{draw\_bonds=None}, \emph{save\_location=None}, \emph{dpi=300}}{}
\end{fulllineitems}



\chapter{Indices and tables}
\label{\detokenize{index:indices-and-tables}}\begin{itemize}
\item {} 
\DUrole{xref,std,std-ref}{genindex}

\item {} 
\DUrole{xref,std,std-ref}{modindex}

\item {} 
\DUrole{xref,std,std-ref}{search}

\end{itemize}



\renewcommand{\indexname}{Index}
\printindex
\end{document}